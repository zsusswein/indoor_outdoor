\documentclass{article}
\usepackage[utf8]{inputenc}
\usepackage{graphicx}
\usepackage{authblk}
\usepackage{soul}
\usepackage[margin=1in]{geometry}
\setlength{\parindent}{0em}
\setlength{\parskip}{1ex}
\usepackage{hyperref}
\usepackage{amsmath}

\usepackage{lineno}
\linenumbers

%\title{Disentangling the seasonality of indoor human activity for airborne disease transmission risk}
\title{Disentangling the rhythms of human activity in the built environment for airborne transmission risk: a large-scale analysis of mobility data}
\author[1,]{Zachary Susswein}
\author[1]{Eva C. Rest}
\author[1,*]{Shweta Bansal}

\affil[1]{Department of Biology, Georgetown University, Washington, DC, USA}
\affil[*]{Corresponding Author: shweta.bansal@georgetown.edu}

\date{March 2022}

\begin{document}

\maketitle

\begin{abstract} %290 words

\noindent \textbf{Background} Since the outset of the COVID-19 pandemic, substantial public attention has focused on the role of seasonality in impacting transmission. Misconceptions have relied on seasonal mediation of respiratory diseases driven solely by environmental variables. However, seasonality is expected to be driven by host social behavior, particularly in highly susceptible populations. A key gap in understanding the role of social behavior in respiratory disease seasonality is our incomplete understanding of the seasonality of indoor human activity.

\vspace{0.05in}
\noindent \textbf{Methods} We leverage a novel data stream on human mobility to characterize activity in indoor versus outdoor environments in the United States. We use an observational mobile app-based location dataset encompassing over 5 million locations nationally. We classify locations as primarily indoor (e.g. stores, offices) or outdoor (e.g. playgrounds, farmers markets), disentangling location-specific visitor counts into indoor and outdoor, to arrive at a fine-scale measure of indoor to outdoor human activity across time and space.

\vspace{0.05in}
\noindent \textbf{Results} We find the proportion of indoor to outdoor activity during a baseline year is seasonal, peaking in winter months. The measure displays a latitudinal gradient with stronger seasonality at northern latitudes and an additional summer peak in southern latitudes. We statistically fit this baseline indoor-outdoor activity measure to inform incorporation of this complex empirical pattern into infectious disease dynamic models. However, we find that the disruption of the COVID-19 pandemic caused these patterns to shift significantly from baseline, and the empirical patterns are necessary to predict spatio-temporal heterogeneity in disease dynamics.

\vspace{0.05in}
\noindent \textbf{Conclusions} Our work empirically characterizes, for the first time, the seasonality of human social behavior at a large scale with high spatio-temporal resolution, and provides a parsimonious parameterization of seasonal behavior that can be included in infectious disease dynamics models. We provide critical evidence and methods necessary to inform the public health of seasonal and pandemic respiratory pathogens and improve our understanding of the relationship between the physical environment and infection risk in the context of global ecological change.

\vspace{0.05in}
\noindent \textbf{Funding} Research reported in this publication was supported by the National Institute of General Medical Sciences of the National Institutes of Health under award number R01GM123007.

\end{abstract}

\section{Introduction}

The seasonality of infectious diseases is a widespread and familiar phenomenon. Although a number of potential mechanisms driving seasonality in directly transmitted infectious disease have been proposed, the causal process behind seasonality is still largely an open question \cite{martinez2018calendar, altizer2006seasonality, grassly2006seasonal}. In the case of the influenza virus, seasonal changes in humidity have been identified as a potential mechanism, with drier winter months enhancing transmission \cite{shaman2009absolute, shaman2010absolute, dalziel2018urbanization}; similar patterns have been observed for respiratory syncytial virus and hand foot and mouth disease \cite{baker2019epidemic, onozuka2011influence}. However, humidity is but one of many mechanisms contributing to seasonality in infectious disease transmission. Seasonal changes in temperature, human mixing patterns, and the immune landscape, among other factors, are thought to contribute to transmission dynamics \cite{metcalf2009seasonality, mossong2008social, kronfeld2021drivers, bakker2021exploring, altizer2006seasonality}. The relative importance of these disparate mechanisms varies across directly-transmitted pathogens and is still largely unexplained \cite{martinez2018calendar, grassly2006seasonal}. The influence of seasonal host behavior on respiratory disease seasonality remains particularly understudied \cite{fisman2012seasonality, kronfeld2021drivers} except for a few notable examples \cite{bharti2011explaining,few2013seasonality, kummer2022measuring}.

%These mechanisms' influence also varies across locations, with the interaction between the landscape of local immunity and that of social mixing playing a key role in modulating seasonality \cite{metcalf2009seasonality, bakker2021exploring}.
For respiratory pathogens spread via the aerosol transmission route, in particular, seasonality may be mediated by multiple behaviorally-driven mechanisms. Aerosol transmission, a significant mode of transmission for a number of respiratory pathogens including tuberculosis, measles, and influenza \cite{tellier2019recognition}, has become increasingly acknowledged during the COVID-19 pandemic \cite{greenhalgh2021ten, wang2021airborne, jayaweera2020transmission, klompas2020airborne, morawska2020time}. The role of aerosols in respiratory disease transmission allows for transmission outside of the traditional 6 ft. radius and 5-minute duration for the droplet mode and implicates human mixing in indoor locations with poor ventilation as being a high risk for transmission, regardless of the intensity of the social contact.
%, with this finding extended to aerosol transmission in the case of the 2003 SARS-CoV-1 pandemic \cite{yu2004evidence, morawska2020time, Conwayremoval}. Consequently, aerosol transmission risk is substantially elevated in indoor, poorly-ventilated locations \cite{kriegel2020predicted, morawska2021paradigm}. 
%Contact rate does not adequately explain seasonal disease transmission dynamics because contact rate remains relatively stable throughout the year, suggesting that seasonality in human mixing patterns could modify exposure to the aerosol transmission route and contribute to seasonality in infectious disease transmission.
While more is known about the spatio-temporal variation in environmental factors such as temperature and humidity in the indoor environment (e.g. \cite{nguyen2016daily}) and the impact these  factors have on airborne pathogen transmission (e.g. \cite{robey2022sensitivity, yang2011dynamics}), limited information is available on rates of human indoor activity and how this varies geographically and seasonally. In the US, most studies quantifying indoor and outdoor time are conducted in the context of air pollutants, suffer from small study sizes, lack spatio-temporal resolution, and are outdated. 
%Employed individuals spent approximately 92\% of their time indoors, 6\% of their time traveling, and only 2\% of their time outdoors (Ott et al, 1989).
The most cited estimates originate from the 1980s-90s and estimate that Americans spend upwards of 90\% of their time indoors \cite{ott1988human}; and more recent data agree with these estimates \cite{klepeis2001national, spalt2016time}. 
%s Americans spend approximately 1.0 hours per day engaged in outdoor activities and 1.2 hours per day engaged in travel (U.S. Bureau of Labor and Statistics, 2019). Daily travel could also occur indoors in a personal vehicle or via public transit, which typically includes time spent in dense, enclosed spaces that provide additional opportunities for contact and disease transmission. Since many workplaces are indoor settings, it follows that the majority of Americans spend more time outside on weekends, but more time in transit on weekdays (U.S. Bureau of Labor and Statistics, 2019). 
While it is well understood that seasonal differences and latitude likely affect time spent indoors, little is known of the spatio-temporal variation in indoor activity beyond this one monolithic estimate, vastly limiting our ability to comprehensively characterize the seasonality of airborne disease exposure risk. 
%One study assessed people’s behaviors outside their home using monthly surveys sent to a random sample of internet users (Lin et al, 2021). However, the spatial resolution of the data is much lower than the mobile-app-based location data used here and most activities included in the survey could be done either indoors or outdoors without further specification. The project does demonstrate a gradual increase in activities outside the home in March and April 2021, approximately when COVID-19 vaccinations became available to the majority of U.S. adults. Working outside the home was the most common activity throughout the pandemic. The largest increases in activities outside the home since March 2021 were visiting friends and going to cafes, bars, and restaurants (Lin et al, 2021). This change in activity is cause for some public health concern, given that indoor dining at restaurants disproportionately impacts transmission because individual visits last longer than at other locations and the activities at a restaurant make it impossible to wear masks consistently throughout the visit (Chang et al, 2021).

Because our understanding of the drivers of seasonality for respiratory diseases has been limited, the modeling of seasonally-varying infectious disease dynamics has been traditionally done using environmental data-driven or phenomenological approaches. Environmental data-driven approaches incorporate seasonality into epidemiological models through environmental correlates of seasonality, such as solar exposure or outdoor temperature \cite{bakker2021exploring, baker2019epidemic, coletti2018shifting}. This  approach to seasonal dynamics controls for inter-seasonal variation in transmission dynamics and measures the strength of correlations between proposed metrics and seasonal variation in force of infection -- although the observed relationship is rarely causally relevant for respiratory disease transmission. In contrast, phenomenological models such as seasonal forcing approaches modulate transmissibility over time without specifying a particular mechanism for this modulation \cite{keeling2001seasonally, altizer2006seasonality}. By applying well-understood functions (such as sine functions), seasonal forcing allows for flexible specification and quantification of dynamics, such as periodicity or oscillation damping, and indirectly captures seasonal variation in non-environmental factors such as school mixing. A significant remaining gap in seasonal infectious disease modeling is thus the ability to empirically incorporate spatio-temporal variation in behavioral mechanisms driving seasonality of disease exposure and transmission.

Thus, despite the role of the indoor built environment in exposure to the airborne transmission route,  seasonal variation in indoor human mixing has not yet been systemically characterized nor integrated into mathematical models of seasonal respiratory pathogens. To address this gap, we construct a novel metric quantifying the relative propensity for human mixing to be indoors at a fine spatio-temporal scale across the United States. We derive this metric using anonymized mobile GPS panel data of visits of over 45 million mobile devices to approximately 5 million public locations across the United States. We find a systematic latitudinal gradient, with indoor activity patterns in the northern and southern United States following distinct temporal trends at baseline. However, we find that the COVID-19 pandemic disrupted this structure. 
%Both exhibit sinusoidal dynamics of $\sigma$ over time with peaks in indoor mixing during the winter, but the southern US has an additional summer peak in indoor mixing. 
Lastly, we fit simple parametric models to incorporate these seasonal activity dynamics into models of infectious disease transmission when indoor activity is expected to be at baseline.
Our work provides the evidence and methods necessary to inform the epidemiology of seasonal and pandemic respiratory pathogens and improve our understanding of the relationship between the physical environment and infection risk in light of global change.

\begin{figure}
    \centering
    \includegraphics[width=0.95\textwidth]{Figure1.pdf}
    \caption{(A) Case studies to highlight varying trends in indoor activity seasonality during 2018: Cook County (in the northern US) has high indoor activity in the winter months and a deep trough in indoor activity in the summer months. Maricopa County (in the southern US) sees moderate indoor activity in the winter and an additional peak in indoor activity during the summer. We apply a 3-week rolling window mean for visualization purposes. (B) A heatmap of the indoor activity seasonality metric for all US counties by week for the calendar year 2018. Counties are grouped by state and are ordered alphabetically by state. We see significant spatio-temporal heterogeneity with distinct trends in the summer versus winter seasons. }
    \label{fig:metric}
\end{figure}

\section*{Results}

Based on anonymized location data from mobile devices, we construct a novel metric that measures the relative propensity for human activity to be indoors at a fine geographic (US county) and temporal (weekly) scale. We characterize the systematic spatio-temporal structure in this metric of indoor activity seasonality with a time series clustering analysis. We also characterize the shift that occurred in the baseline patterns of indoor activity seasonality during the COVID-19 pandemic. We note that this seasonal variation in the propensity of human activity to be indoors differs from the variation in overall rates of contact between individuals, which does not vary seasonally \ref{fig:contactnotseasonal}.
%, we find this systematic structure is well-described by three clusters -- corresponding to major geographic regions of the United States. W
Lastly, we fit non-linear models to the indoor activity metric at baseline, comparing the ability of a simple model to capture seasonal variation in transmission risk. 
%We apply these fitted parameter values to a model of infectious disease dynamics, demonstrating the implications of empirically-derived correlates of seasonality for infectious disease modeling.

\subsection*{Quantifying empirical dynamics in indoor activity}
The indoor activity seasonality metric, $\sigma$, captures the relative frequency of visits to indoor versus outdoor locations within an area. The components of $\sigma$ capture the degree to which indoor and outdoor locations are occupied;  when $\sigma = 1$, a given county is at its county-specific average propensity (over time) for indoor activity relative to outdoor. When $\sigma < 1$, activity within the county is more frequently outdoor and  less frequently indoor than average, while $\sigma > 1$ indicates that activity is more frequently indoor and less frequently outdoor than average. Thus, a $\sigma$ of 1.2 indicates that the county's activity is 20\% more indoor than average and a $\sigma$ of 0.80 indicates that the county's activity is 20\% less indoor than average (additional details in methods).

Through this metric, we measure the relative propensity for human activity to be indoors for every community (i.e. US county) across time (at a weekly timescale), finding significant heterogeneity between counties (Figure \ref{fig:metric}A). The representative examples of Cook County, Illinois (home of the city of Chicago in the midwestern US) and Maricopa County, Arizona (home of the city of Phoenix in the southwestern US) highlight systematic spatial and temporal heterogeneity in indoor activity dynamics. In Cook County, indoor activity varies over time, at its peak in the winter, with the relative odds of an indoor visit well above average. During the summer, $\sigma$ in Cook County reaches its trough, with activity systematically more outdoors on average. On the other hand, the variation of $\sigma$ across time in Maricopa County is characterized by a smaller winter peak in indoor activity, and an additional peak in the summer (i.e. July and August); this peak occurs concurrently with the trough in Cook County. Unlike in Cook County, $\sigma$ in Maricopa County is lowest in the spring and fall. These representative counties illustrate the systematic within-county variation in indoor activity over time, as well as the between-county variation in temporal trends as represented in Figure \ref{fig:metric}B for all US communities.

\begin{figure}
    \centering
    \includegraphics[width=0.95\textwidth]{Figure2.pdf}
    \caption{Using a time series clustering approach on the indoor activity time series for each US county, we identify groups of counties that experience similar trends in indoor activity. Locations in the northern cluster (light blue) follow a single peak pattern with the highest indoor activity occurring every winter. Locations in the southern cluster (dark blue) experience two peaks in indoor activity each year, one in the winter and a second, smaller one in the summer. The third cluster also experiences two peaks not matching environmental conditions, but potentially corresponding to winter tourism areas. We apply a 3-week rolling window mean to the time series for visualization purposes.}
    \label{fig:clustermap}
\end{figure}

To identify systematic geographic structure, we cluster the heterogeneous time series of county-level, weekly indoor activity. We find three geographic clusters corresponding to groups of locations that experience similar indoor activity dynamics (Figure \ref{fig:clustermap}). These clusters primarily split the country into two clusters: a northern cluster and a southern cluster. Among the communities in the northern cluster, activity is more commonly outdoor over the summer months, trending toward indoor during fall, with a peak in the winter months, as observed in Cook County. Comparatively, the southern cluster has a larger winter peak (i.e. between December and February) and a smaller summer peak (i.e. between July and August); most summer peaks are less extreme than that of Maricopa County (shown). %In this cluster, the contacts are most frequently outdoor in the spring, between April and early July. 
We hypothesize that these two clusters are consistent with climate zones. We compare these clusters to climate zones defined for the construction of the indoor built environment and find that there is substantial consistency between the two (Supplementary Figure \ref{fig:iecc}).
The third cluster differs substantially: it is geographically discontiguous and its two annual peaks occur during the spring (close to April) and fall (closer to November) seasons. Thus, the counties in this cluster have outdoor activity more frequently than average during both the winter and the summer. The counties in this cluster correspond to locations that are hubs for winter tourism, which we speculate is driving their unique dynamics (Supplementary Figure \ref{fig:ski}).
%Overall, despite the substantial heterogeneity in $\sigma$, we find that this temporal heterogeneity falls into three distinct geographic groups. 
%The correspondence between the identified clusters and the climate zones suggests a shared climate-based mechanism behind the observed geographic structure in temporal variation in $\sigma$ and variation in the built environment in which these indoor contacts occur.

\begin{figure}
    \centering
    \includegraphics[width=0.75\textwidth]{Figure4.pdf}
    \caption{Indoor activity during the COVID-19 pandemic was shifted: We compare indoor activity trends in the baseline years of 2018 and 2019 to the pandemic year 2020 in four case study locations. We find that most locations saw a shift in their indoor activity patterns, while others (such as Maricopa County) did not. We also find that while overall activity was diminished uniformly during the Spring of 2020, indoor activity decreased in some locations (Travis County, Texas and Baltimore County, Maryland) and increased in others (Charleston County, South Carolina). We apply a 3-week rolling window mean to the time series for visualization purposes.}
    \label{fig:2020indoor}
\end{figure}

\subsection*{Characterizing pandemic disruption to baseline indoor activity seasonality}
In addition to the description of indoor activity seasonality at baseline, we examine the impact of a large-scale disruption -- the COVID-19 pandemic -- to these patterns. We compare indoor activity seasonality during the COVID-19 pandemic in 2020 to the baseline patterns of 2018 and 2019. We find that the temporal trends in indoor activity are less geographically structured in 2020 than those of previous years (see Supplementary Figure \ref{fig:2020clustermap} for a characterization of the time series patterns). We find that in most locations indoor activity deviated from pre-pandemic trends. We focus on four case studies to highlight the varying impacts on indoor activity of the pandemic disruption (Figure \ref{fig:2020indoor}). In all four communities, 2020 indoor activity trends shift from 2018 and 2019 patterns, with Maricopa County (home of the city of Phoenix, AZ) showing the least perturbation relative to prior years. We also find that in early 2020, when there was substantial social distancing in the United States (e.g. school closures, remote work), activity was more likely to be outdoors than in prior years, independent of changes in overall activity levels. With our case studies, we highlight that social distancing policies can have different impacts on airborne exposure risk in different locations: while some locations, such as Travis County (home of Austin, Texas), shifted activities outdoors during this period, reducing their overall risk further, other locations, such as Charleston County, South Carolina (home of Charleston, South Carolina) increased indoor activity above the seasonal average during this period, potentially diminishing the effect of reducing overall mobility.

%Our findings for 2020 also highlight that the effect of social distancing policies can vary depending on the indoor contact context of a community
%The trend in sigma is independent of contact *number* -- in 2020 lots of social distancing but not always a resulting change in sigma
%Empirically, local changes in sigma varied over the course of the pandemic independently of contact with some places increasing outdoor contact in response to pandemic guidance (Boston); and others resulting in more indoor contact during intense social distancing periods (Charleston)
%We investigated the disruption to mobility patterns during the COVID-19 pandemic by comparing 2020 data to baseline mobility patterns in 2018 and 2019. 
%Overall mobility declined during the COVID-19 pandemic across geographic regions, particularly during early months of the pandemic in 2020, likely due to school closures, remote work and other social distancing measures. In contrast, the proportion of indoor contacts during the same period displayed notable spatial heterogeneity. The proportion of indoor contacts remained relatively stable before and during the COVID-19 pandemic in Maricopa County, Arizona and Charleston County, South Carolina. However, the proportion of indoor contacts declined in Baltimore County, Maryland and Travis County, Texas during that time period, indicating a shift toward outdoor contacts. Full understanding of disease transmission risk and implementation of effective interventions thus requires knowledge not only of the number of contacts occurring in a given county, but also where contacts are likely to occur.

\begin{figure}
    \centering
    \includegraphics[width=0.8\textwidth]{Figure3.pdf}
    \caption{(A) Sine curves fit to the 2018 and 2019 time series data (analogous to seasonal forcing model components) fit the northern cluster better than the southern cluster, with a markedly poorer fit for the southern cluster's second summer peak.  (B) Regional seasonal forcing models display variation in patterns of disease incidence omitted by a non-seasonal model, but even region-level seasonal forcing does not fully capture within-cluster county-level variation.}
    \label{fig:sinfit}
\end{figure}

\subsection*{Implications for modeling seasonal disease dynamics}
We use this finely-grained spatio-temporal information on indoor activity to incorporate airborne exposure risk seasonality into compartmental models of disease dynamics using common, coarser seasonal forcing approaches. To investigate the impact of heterogeneity in $\sigma$ on the estimation of seasonal forcing for infectious disease models, we fit a sinusoidal model to the time series of indoor activity for each of the primary clusters (Figure \ref{fig:sinfit}A). We find that the parameters of seasonality vary across clusters: the amplitude is higher, and phase is lower in the northern cluster compared to the southern cluster, indicating a difference in the variability of indoor and outdoor activity seasonality in each cluster (Supplementary Figure \ref{fig:sineparamters}). The sinusoidal model was a poorer fit for the southern cluster, particularly around the second peak of indoor activity during the summer months. These differences in best fit indicate that sinusoidal models may have an overly restrictive functional form, limiting the accuracy of the approximation, and may underestimate the impacts of seasonality on transmission,  obscuring systemic differences between regions. Furthermore, differences in seasonal activity of the observed magnitude can have important implications for disease modeling; applying region-level and county-level forcing to a simple disease model alters incidence patterns (Figure \ref{fig:sinfit}B). Although region-level seasonality changes incidence timing and peak size relative to a non-seasonal model, it does not fully capture the changes produced by county-level seasonality. These differences indicate that while coarser geographic approximations of seasonality can be appropriate, these approximations can also oversimplify, reducing the accuracy of disease models.   Additionally, while simple models of baseline indoor activity can capture seasonality in exposure risk, disruptions such as pandemics can alter this baseline structure and increase heterogeneity. 


\section*{Discussion}
%NOTES ON MECHANISMS:
%The indoor activity metric may capture increased risk of transmission by affecting host susceptibility, transmissibility or contact:
%Increased indoor density indicates increased or longer duration airborne contact between susceptible and infected individuals
%Under homogeneous-mixing within locations, increased indoor density may indicate increased droplet contact between susceptible and infected individuals
%Increased indoor activity might indicate higher susceptibility as poor ventilation, increased pollutants, reduced solar exposure, and low humidity of indoor environment has been shown to weaken immune response [NEED REFS]
%Increased indoor activity might indicate higher exposure as low humidity due to HVAC in indoor environments has been shown to increase viral survival and HVAC re-circulation has been shown to increase viral dispersion \cite{lu2020covid, liao2005probabilistic}.
%Respiratory pathogen seasonality can be viewed as a way to mitigate viral spread during low-transmission periods or prepare vaccination and intervention strategies to prevent illness during seasonal peaks. However, 
The seasonality of influenza, SARS-CoV-2, and other respiratory pathogens depends not only on environmental variables but also on the social behavior of hosts. In settings with little prior immunity -- such as a pandemic -- host social behavior (generating contacts during which transmission may occur) primarily drives heterogeneity in disease dynamics, and seasonality is dwarfed by susceptibility \cite{baker2020susceptible}. In settings with higher rates of immunity, contact remains critically important, and seasonal changes in contacts (both direct and indirect) can contribute to movement of $R_{t}$ above and below 1 -- providing noticeable changes in incidence. Although environmental variables play a role in the seasonality of respiratory pathogens, the role of host social behavior in pathogen seasonality is poorly understood, driven by a poor understanding of indoor versus outdoor social interactions and interactions between behavior and the environment. In this study, we propose a fine-grain measure of indoor activity seasonality across time and space. We determine that indoor activity seasonality displays significant spatio-temporal heterogeneity and that this variability can be decomposed into two geographic groups representing distinct temporal dynamics in indoor activity. We also find that while indoor activity seasonality may be highly predictable under baseline conditions, disruptions such as the COVID-19 pandemic can alter these patterns. Finally, we provide an illustration of how our findings can be incorporated into classical infectious disease models using parsimonious models of exposure seasonality.

%The potential impact of indoor versus outdoor social interactions on transmission stems from the assumption that the primary mode of COVID-19 transmission is contact with infectious virions via respiratory or aerosolized particles. That COVID-19 spreads via respiratory transmission is not in question, and has led to the implementation of mask-wearing as an important preventative measure. However, some debates persist concerning specific aspects of transmission. Some studies assert that small aerosolized particles, which can remain infectious and airborne for long periods of time, are a key mode of transmission (Coleman et al, 2021; Buonanno et al, 2020). This supports the role of asymptomatic transmission from COVID-19 positive individuals without respiratory symptoms. Other studies include larger respiratory droplets, which drop out of the air more quickly and may be associated with acute respiratory symptoms like coughing and sneezing (Bu et al, 2021; Bazant et al, 2021). In addition, some studies suggest that the size of the infectious particle determines how far into the respiratory tract it can go, creating a higher risk of infection if small particles reach farther into the lungs and bypass host innate immune responses (Chu et al, 2020). Fundamentally, a confluence of multiple factors, including behavioral and environmental traits, determine COVID-19 transmission.
The indoor activity seasonality that we quantify may reflect heterogeneity in transmission risk via a number of mechanisms including those affecting host contact, susceptibility, or transmissibility. %The possibility of asymptomatic transmission reduces the likelihood of public health measures such as isolation and quarantine fully inhibiting mixing of susceptible and infected individuals in one location. 
Increased indoor activity may indicate longer-duration airborne contact (e.g., co-location without direct interaction) between susceptible and infected individuals, elevating respiratory transmission risk. Increased indoor density may also suggest increased droplet contact (e.g., a conversation in close proximity), under homogeneous mixing. Additionally, indoor activity may suggest increased susceptibility as poor ventilation, increased pollutants, reduced solar exposure, and low humidity of the indoor environment have been shown to weaken immune response \cite{moriyama2020seasonality}. Finally, increased indoor activity may indicate an increase in transmissibility due to higher exposure as low humidity caused by climate control (heating, ventilation, and cooling, HVAC) in indoor environments has been shown to increase viral survival and HVAC re-circulation has been shown to increase viral dispersion \cite{lu2020covid, liao2005probabilistic}.
%because indoor locations lack many environmental factors associated with reduced pathogen transmission, including adequate ventilation, exposure to UV light, diminished concentrations of air pollutants, and higher humidity levels. 
%Low humidity, commonly caused by HVAC systems in indoor environments, have been shown to increase viral survival, leading to a higher likelihood of pathogen exposure indoors. HVAC systems also recirculate air in indoor environments, increasing the dispersion of aerosolized viral particles \cite{lu2020covid, liao2005probabilistic}. Purposeful changes to indoor locations may be necessary to address these risks and reduce respiratory transmission in situations where high indoor density is unavoidable.
While our new measure does not disentangle these component mechanisms, it represents an integrated seasonality in exposure risk due to all of these factors and can help lead us to a more complete understanding of the heterogeneity in disease dynamics and outcomes.

We find that spatio-temporal heterogeneity in the indoor activity metric can be classified into two large geographically-contiguous groups in the northern and southern United States. These groups closely correspond to built environment climate zones, potentially explaining this systematic variability. We note, however, that while these clusters overlap with climate classifications, this correspondence does not suggest that environmental variables such as temperature and humidity should be used to represent behavioral heterogeneity. Climatic factors within these climate zones may be related to, but not necessarily correlated with, the seasonality of human mixing within these zones. Additionally, even in the case that environmental factor variability drives behavioral variability, it would be critical to capture the effect of behavior on disease directly so as to not obscure any direct effects of climatic factors on disease.
%This result does not provide any evidence for environmental variables driving seasonality in disease transmission dynamics -- though it may occur. 

%Given the significant ways in which indoor activity can affect exposure risk profiles, it would be important to incorporate seasonality in exposure risk to future models of disease dynamics. 
%In particular, our understanding of the seasonality of respiratory viruses such as influenza, RSV and now SARS-CoV-2 remains limited, and the addition of an empirically-driven understanding of temporal heterogeneity in behavior may prove productive.
We illustrate how to incorporate seasonality in exposure risk to future models of disease dynamics using a simple phenomenological model. We use this traditional model of infectious disease dynamics to evaluate the implications of the spatial coarseness of seasonal forcing. Our results suggest that the substantial local heterogeneity in the dynamics of indoor activity across time and space could be large enough to alter seasonality in infectious disease dynamics. While our work does not consider observed transmission patterns, we suggest that researchers carefully consider the spatial scale on which they model seasonality in theoretical models, commonly used for scenario analysis and model-based intervention design (e.g., \cite{borchering2021modeling}). We additionally highlight that the use of simple or complex functional forms of seasonality requires statistical fits to baseline data, and in the case of disruptions, these fitted models may no longer be appropriate. As we show, patterns of human mobility changed substantially during the COVID-19 pandemic, potentially contributing to changes in infectious disease seasonality.
%, the use of a simple seasonal forcing approach may be  model is warranted for some locations such as the ones in the northern cluster identified in our work. For locations with more complex seasonal dynamics (such as the communities in the southern cluster we identify which have a larger winter peak as well as a smaller summer peak in indoor activity), more complex statistical approaches (e.g Fourier transforms) may be necessary.
%While seasonality has been commonly associated with seasonal influenza transmission, this metric may help explain patterns that do not match with traditional seasonality models or emerging epidemics that do not have established seasonal variation.
%Seasonality is perhaps most commonly associated with seasonal influenza transmission, but also plays an important role in COVID-19 transmission and respiratory disease transmission in general. Combining the impacts of seasonality with latitude can substantially impact disease dynamics. Across both hemispheres, seasonal influenza prevalence peaks in winter months (Almeida et al, 2018). Influenza A seems to be more latitude-dependent than Influenza B, which is most prevalent in colder months regardless of latitude. By contrast, influenza A prevalence tends to peak in January or February for higher latitude locations and April through June for lower latitude locations (Yu et al, 2013).

Recent work during the COVID-19 pandemic demonstrates the impact of reduced occupancy in indoor locations and increasing outdoor activity on the likelihood of disease transmission. In particular, behavioral interventions or nudges that reduce occupancy are more impactful than reducing overall mobility as they reduce visitor density and the likelihood of density-dependent airborne transmission.
%since they prevent busy periods when high visitor density increases transmission risk. This approach is also economically favorable, allowing businesses with fewer visitors or with periods of lower visitor density to remain open with reduced transmission risk \cite{chang2021mobility}. 
%In densely populated urban areas, reduced mobility was associated with a decline in case rates, such as the restrictions enacted during strict lockdown protocols. 
Similarly, the availability of outdoor areas in urban settings, such as public parks, has been demonstrated to reduce case rates when population mobility becomes less restricted \cite{johnson2021landscape}.
Our results suggest that such public health strategies should be implemented in a targeted manner, informed by real-time data and with clear communication of the goals.
We found notable changes occurred in indoor activity seasonality at the start of the COVID-19 pandemic, despite relatively consistent patterns during the spring season in prior years. Designing a behavioral strategy and measuring its effectiveness without real-time data could thus be misleading.
%An policy based on seasonality in indoor activity based on pre-pandemic data would have been unable to the dramatic changes observed in Charleston County.
%Local environmental factors also necessitate individualized policies and plans for resource allocation. Maricopa County, with two peaks in indoor behavioral seasonality, might require different strategies and dual resource allocation because the peaks occur at vastly different times of year. The indoor behavioral seasonality peak in the winter is similar to many other counties at more northern latitudes, so may benefit from similar intervention strategies. However, the summer peak is distinct from the northern cluster and may require distinct interventions, like access to water sources, shaded outdoor spaces, or socially distanced cooling centers that would not be effective in all areas. Localized interventions and specific or time-sensitive resource allocations are necessary to maximize reduction in transmission risk through public health policies.
Our finding of two distinct geographic clusters of indoor activity suggests the need for geographical targeting of strategies to reduce indoor transmission risk. While northern latitudes might benefit from decreased indoor occupancy and increased outdoor activity in Northern Hemisphere winters, southern latitudes should be additionally targeted for such interventions in the summer months.
%Efforts to reduce transmission risk through social distancing or other public health policies could be more effective if policies are tailored to local activity patterns and are specific to contacts posing the highest risk of transmission. Social distancing policies were potentially impactful in areas where both overall and indoor activity decreased in spring 2020, such as Baltimore County, Maryland. By contrast, areas like Charleston County, South Carolina saw a decrease in overall activity, but an increase in indoor activity, diminishing any positive effects of contact-limiting public health policies. If contacts are reduced, but the contacts that continue to occur are indoors or in high-risk environments, transmission risk may not be sufficiently reduced. Policies designed to specifically target indoor contacts, rather than overall activity, may thus be more effective at meaningfully reducing disease transmission. 
Lastly, our findings highlight the need to communicate the goals of behavioral interventions clearly. While all communities universally reduced overall activity during the early days of the COVID-19 pandemic, some increased indoor activity during this time, potentially diminishing the positive effects of the social distancing policies put into place. A public health education campaign to clarify the role of indoor interactions in transmission risk may have ameliorated this.

%Role of behavior seasonality and local quirks in timing and peak size (as it translates to increased exposure risk) needs to be taken into account for intervention targeting and for resource allocation.
%social distancing != exposure risk if you don't say what that means, if you decrease overall contact but increase indoor, you may not be doing anything helpful
%need to do distancing in local ways because of unique environmental 
Our study leverages a novel data stream made available to researchers due to the COVID-19 pandemic. Similar datasets are available globally, part of a \$12 billion location intelligence industry \cite{keegan2021there}. Such novel data streams offer many opportunities to address long-unanswered questions in infectious disease and climate change behavior dynamics, but these data must be interpreted carefully. Safegraph's mobile-app-based location data does not include data on individuals less than 16 years of age \cite{privacy}. While we may expect that children under 12 may be accompanied by adults that may be represented in the dataset, our metric likely does not capture the activity dynamics of older children (children 12-15 make up 5\% of the US population).  
For those included in the Safegraph database, representation is dependent on smartphone usage and a number of business processes not transparent to users of the data, thus we expect that there is geographic variation in the representativeness of the data. Smartphone ownership has increased in recent years, with 85\% of US adults reporting smartphone ownership; however, smartphone usage does vary significantly by age, with only 61\% of adults over 65 reporting smartphone use \cite{smartphone_usage}. Additionally, data shows that location sharing among mobile users is not significantly biased by age, gender, race/ethnicity, income or education (with 40-65\% of all demographic groups participating in location sharing) \cite{zickuhr201128}.
Based on an analysis done by Safegraph, the panel is representative of race, educational attainment, and income \cite{safegraph_bias}. On the other hand, a recent independent analysis shows that older and non-white individuals are less likely to be captured in the panel for POI-specific analyses \cite{coston2021leveraging}. It is important to note that both studies are associative in nature as the devices in the panel are fully anonymized, so no device-level demographic data exists. Continued work to understand the sampling biases of such datasets will be needed so that improved bias correction approaches can be developed \cite{coston2021leveraging}.
%Geographic areas with lower device counts or fewer locations tracked in the data set, like rural areas, may be underrepresented. In contrast, densely populated areas or large businesses may be overrepresented in this data set. 
%Populations that do not widely use mobile phones are not reflected in the data set, particularly young children, older populations, and other demographics with lower mobile phone ownership. As children are important routes of transmission for some respiratory pathogens through schools and group activities, the lack of mobility data for young children presents limitations for some public health applications. Smartphone ownership increased from 42\% of U.S. adults over the age of 65 in 2017 to 61\% in 2021, but still lags behind younger demographics. Smartphone use is also higher in more educated and higher income populations. 
%but could miss within-county heterogeneity in socioeconomic status, required mobility during the pandemic, and other determinants of population health (Riva et al, 2009; Chang et al, 2021). Additionally, this type of data does not reflect some important factors in the likelihood of respiratory disease transmission, including face-to-face interactions, social contacts within a given setting, or length of time spent in a given location (Naud et al, 2020). Data is also limited to mobility outside of the home, omitting household contacts or non-household member contacts in residential locations.
%[Caveats to add: outside home; unclear locations; adults only; demographic bias; visit is anything > 1 minute; not including dwell time]
Additionally, we limit our scope in this study to consider only the number of visits and do not incorporate information about visit duration. The dataset counts all visits of one minute or longer. For disease transmission, there may be a threshold duration required for an interaction between an infected and susceptible individual for infection to be propagated. These thresholds are not well-understood for all respiratory diseases, but evidence that SARS-CoV-2 transmission can occur with brief encounters has emerged \cite{pringle2020covid}. While the Safegraph dataset does provide median dwell times for POIs, the likely significant heterogeneity in the distribution of dwell times remains unknown and is difficult to capture in an aggregated manner.%Thus, we acknowledge the numerous limitations of our data source and study. Our goal with this work is to present an improvement on the state-of-the-art on behavioral seasonality for airborne transmission risk, and urge that our findings be interpreted qualitatively.

Our metric and analysis also focus on the US county scale to reflect the finest scale generally used for infectious disease modeling as well as public health decision-making. This choice is likely to ignore some within-county heterogeneity and means that our metric does not represent the experience of all groups, particularly by socioeconomic status. For example, low-income and racially marginalized communities have systematically less access to outdoor, natural spaces and spend more time indoors due to structural inequities including lack of paid leave \cite{spalt2016time,nesbitt2019green,sefcik2019nature}. Such socio-economic disparities have been further exacerbated during the pandemic, which potentially affects our indoor activity estimates during 2021. Thus our estimate of a county's indoor transmission risk may represent an underestimate of the risk experienced by individuals in these communities. We commit to continued work to better characterize the transmission risk experienced by vulnerable populations.
Lastly, we acknowledge that data modeling work that can influence public health policy decisions, particularly during an ongoing crisis, must be done with care to prevent misconceptions from having adverse effects on risk perception and policies \cite{carlson2020weather}. We thus strongly note that while our measure of indoor behavioral seasonality provides a potential driver of respiratory disease seasonality, it remains one among many complex factors which integrate to predict the transmission potential of an ongoing epidemic or pandemic \cite{susswein2021ignoring}. Thus we cannot rely on behavioral seasonality to diminish transmission naturally, and pandemic intervention strategies should not be planned around behavioral seasonality while population susceptibility remains high in so many locations. 
%In particular, our understanding of the seasonality of respiratory viruses such as influenza, RSV and now SARS-CoV-2 remains limited, and the addition of an empirically-driven understanding of temporal heterogeneity in behavior may prove productive.
%While we argue here that seasonality of indoor human activity can help explain seasonality in infectious disease dynamics, we do not claim that such seasonality is enough to control SARS-CoV-2 transmission alone. Rather, we suggest that aerosol pathogen transmission could be mediated through this mechanism (i.e. increasing effective number of contacts), enhancing transmission dynamics. In settings where transmission is a concern, such as control of a pandemic disease,
%environmental determinants of transmission are almost certainly outweighed by the lack of population immunity, like in the case of SARS-CoV-2 \cite{carlson2020weather}. 
%Future analyses disaggregating visits by socioeconomic group may reflect differences in occupation type and setting by socioeconomic status. Previous research demonstrates time spent indoors or outdoors differs by racial group, income, and education level \cite{spalt2016time}. During the COVID-19 pandemic, low-wage workers and workers from low-income households constituted a disproportionate number of essential workers with higher likelihood of COVID-19 exposure \cite{wolfe2021inequalities} \cite{mccormack2020economic}. These occupation types and settings may increase indoor contacts for low-wage workers, particularly in the service industry. By contrast, low-wage agricultural workers may have increased outdoor contacts, although the nature of POIs in this data set may under-represent agricultural work sites. By aggregating mobility patterns at the county level and across socioeconomic groups, demographic differences in contact seasonality patterns may be missed, leading to flawed inferences about behavior or transmission risk.
%[Future: socio-economic disaggregation \cite{spalt2016time}]

Ongoing global change events highlight the importance of this work, as it informs how widespread disruptions may shift patterns of indoor activity, potentially altering traditional infectious disease seasonality. 
%Public health measures to reduce the likelihood of indoor respiratory transmission, including mask mandates in indoor locations and investments in HVAC infrastructure. These measures are particularly important in locations with high indoor density and could even be implemented on an as-needed basis when transmission likelihood is highest, as measured by the indoor contact seasonality metric. Our metric could also inform estimates of environmental pollution exposure, allowing community health projects to focus environmental air monitoring efforts on high-need areas at high-risk times of year when indoor contact peaks. 
Climate change events will continue to cause significant disruption to normal behavior patterns; mechanistic understanding of infectious disease seasonality and real-time data collection will be crucial components of future disease control efforts. While other global change events may impact indoor activity in different ways than the COVID-19 pandemic, a rigorous understanding of the impact of host behavior on infectious disease allows policymakers and emergency preparedness experts to effectively address future disruptions.

%Implications:
%\begin{itemize}
   % \item Our study informs need for mask mandates, HVAC infrastructure investments
    %\item Informs environmental pollutant exposure
    %\item Informs shift in indoor activity due to a disruption which has implications for global change events
%\end{itemize}


%%%%%%%%%%%%%%%%%%%%%%%% METHODS
\section*{Methods}

\subsection*{Data Source}
We use the SafeGraph Weekly Patterns data, which provides foot traffic at public locations (``points of interest", referred to as POIs from here on) across the US based on the usage of mobile apps with GPS \cite{safegraphpatterns}. The data are from 2018 to 2020, and 4.6 million POIs are sampled in all years of our study.  The data is anonymized by applying noise, omitting data associated with a single mobile device, and is provided at the weekly temporal scale. Data are sampled from over 45 million smartphone devices (of approximately 275-290 million smartphone devices in the US during 2018-2021 \cite{smartphone}), and does not include devices that are out of service, powered off, or ones that opt out of location services on their devices.

As a data cleaning step, we use spatial imputation for any county-weeks where sample sizes are small. For locations in which the visitor count is less than 100, we impute their indoor activity seasonality using a weighted average of $\sigma$ in the neighboring locations (where neighbors are defined based on shared county borders).

Ethical review for this study was sought from the Institutional Review Board at Georgetown University and the study was approved on October 14, 2020.

%The CBG is the highest resolution for census demographic information and generally contains between 600 to 3,000 people. Each POI in the SafeGraph database includes information on the number of devices that enter the POI on an hourly basis, a distribution of dwell times, and the device’s home CBG. SafeGraph determines the device’s home CBG by analyzing 6 weeks of data during nighttime hours (between 6 pm and 7 am). While not all devices can be assigned to a home location, a majority of the mobility data includes a home location. The Weekly Patterns data is used to define arrival and dwell time distributions to determine if people come into contact at POIs.

\subsection*{Defining indoor activity seasonality}
Safegraph Points of Interest (POIs) are locations where consumers can spend money and/or time and include  schools, hospitals, parks, grocery stores, and restaurants, etc, but do not include home locations. Each POI is assigned a six-digit North American Industry Classification System (NAICS) code in the SafeGraph Core Places dataset to classify each location into a business category. We classify each NAICS category as primarily indoor (e.g. schools, hospitals, grocery stores), primarily outdoor (e.g. parks, cemeteries), or unclear if the location is a potentially mixed indoor and outdoor setting. Approximately 90\% of POIs were classified as indoors, 6.5\% were classified as outdoors, and 3.5\% were classified as unclear.

%It is important to keep in mind that SafeGraph data does not capture every device and not all devices are linked to a home CBG. In the US, there are approximately 260.2 million smartphones users [27]. The data provided by SafeGraph represents approximately 6-8\% of those devices, based on the data released between February and June of 2020. Devices that are out of service, not moving, lack a tracking app, or have opted out of location services are not included in the data. While the SafeGraph data includes a subset of the true population, the data includes the same places and individuals across time (to the extent possible) and the data has been analyzed to ensure it is representative of population demographics [28].
%Device-registered visits are rescaled to reflect a larger percent of the general population by accounting for the number of people that each device represents. Fig 2 illustrates how the population per device varies across CBGs in Bernalillo County using data from February, 2020. Using the SafeGraph data that links device-registered visits to a home CBG, hourly device count at each POI is rescaled to estimate the number of visits. In Bernalillo County, 75\% of visits are associated with a home CBG. Visits lacking an assigned CBG are scaled using the county average population per device (18.7 people per device).

We define $\widetilde{\sigma}_{it}$, equation (1), as the propensity for visits to be to indoor locations relative to outdoor locations. We aggregated raw visitor counts, defined when a device is present at a non-home POI for longer than one minute, to all indoor POIs and all outdoor POIs in a given week ($t$) at the U.S. county level ($i$). Visitor counts are normalized by the maximum visitor counts for indoor or outdoor locations in each county during the year 2019.
%Since $\sigma_{it}$ is computed as the proportion of indoor to outdoor visits, a value of $\sigma > 1$ indicates contacts are more likely to occur indoors while a value of $\sigma < 1$ indicates contacts are more likely to occur outdoors.
\begin{equation}\label{eq1}
\widetilde{\sigma}_{it} = \frac{N^{indoor}_{it} / max_t \{N^{indoor}_{it}\}}{N^{outdoor}_{it} / max_t \{N^{outdoor}_{it}\}}\end{equation}
This metric is then mean-centered to arrive at a relative measure of indoor activity seasonality, ${\sigma}_{it}$, which is comparable across all counties:
\begin{equation}\label{eq2}
{\sigma}_{it} = \frac{\widetilde{\sigma}_{it}}{\mu_{\tilde{\sigma}}}\end{equation}
We note that $\mu_{\tilde{\sigma}}$ is not spatially structured (see Supplementary Figure \ref{fig:indooroutdoormean}).

\subsection*{Time series clustering analysis}
To characterize groups of US counties with similar indoor activity dynamics, we use a complex networks-based time series clustering approach \cite{rosensteel2021characterizing}. We first calculate the pairwise similarity between z-normalized indoor activity time series for each pair of counties, $i$ and $j$ using the Pearson correlation coefficient ($\rho_{ij}$). For pairs of locations where $\rho_{ij} \geq 0.9$, we represent the pairwise time series similarities as a weighted network where nodes  are US counties and edges represent strong time series similarity.

We then cluster the time series similarity network using community structure detection. This method effectively clusters nodes (counties) into groups of nodes that are more connected within than between. The resulting clustering thus represents a regionalization of the U.S. in which regions consist of counties that have more similar indoor activity dynamics to each other than to other regions. One benefit of the network-based community detection approach over traditional clustering methods is that community detection does not require user specification of the number of clusters (regions, in this case); instead the number of clusters emerge organically from the data connectivity \cite{clustering}. For community detection, we use the Louvain method \cite{blondel2008fast}, a multiscale method in which modularity is first optimized using a greedy local algorithm, on the similarity network with edge weights (i.e. time series correlations) using a \texttt{igraph} implementation in \emph{Python} \cite{louvain_github_igraph}.

\subsection*{Disruptions to indoor activity due to pandemic response}
We investigate the COVID-19 pandemic's impact on indoor activity seasonality by comparing pre-pandemic mobility patterns in 2018 and 2019 with mobility patterns during the COVID-19 pandemic in 2020. We compared the proportion of indoor visitor counts at the county level, $\sigma_{it}$, across 2018, 2019, and 2020 to examine changes in indoor activity seasonality during the COVID-19 pandemic. We also examined total activity, aggregating visitor counts to indoor, outdoor, and unclear POIs by week and mean-centering them for each US county during the COVID-19 pandemic in 2020.

\subsection*{Incorporating indoor activity into infectious disease models}
We seek to illustrate the impact of incorporating seasonality into an infectious disease model using a phenomenological model versus empirical data. To achieve this, we parameterize a simple compartmental disease model with a seasonality term, using either our empirically-derived indoor activity seasonality metric or an analytical phenomenological model of seasonality fit to this metric.

\subsubsection*{Phenomenological model of seasonality}
We first fit our empirically-derived indoor activity seasonality metric using a time-varying non-linear model. We specify the time-varying effect as a sinusoidal function as is commonly done to incorporate seasonality into infectious disease models phenomenologically. The indoor activity seasonality, $\sigma_{it}$ for cluster $i$ at week $t$ is specified as:
$\sigma_{it} = 1 + \alpha_i \sin(\omega_i t + \phi_i)$, where $\alpha_i$ is the sine wave amplitude, $\omega_i$ is the frequency and $\phi_i$ is the phase. We fit a model for locations in the northern cluster separately from those in the southern cluster, as identified above. We fit the parameters for this model using the $\texttt{nlme}$ package in $\texttt{R}$.

\subsubsection*{Disease model}
We model infectious disease dynamics through a simple SIR model of disease spread:
\begin{gather*}
    \frac{dS}{dt} = -\beta_0\beta(t) SI\\
    \frac{dI}{dt} = \beta_0 \beta(t) SI - \gamma I\\
    \frac{dR}{dt} = \gamma I\\
\end{gather*}

We incorporate alternative seasonality terms to consider the impact of heterogeneity in indoor seasonality on disease dynamics. For the northern and southern cluster separately, we define modeled seasonality as $\beta(t) = 1 + \alpha \sin(\omega t + \phi)$, with the fitted parameters for each cluster (Supplementary Figure \ref{fig:sineparamters}). We also consider two exemplar locations for empirical estimates of seasonality, where $\beta(t) = \sigma_t$ after rolling window smoothing: Cook County for an example county from the northern cluster, and Maricopa County for an example location from the southern cluster. We also compare against a null expectation where $\beta(t) = 1$. (All seasonality functions are illustrated in Supplementary Figure \ref{fig:betafunction}). We assume that $\beta_0 = 0.0025$ and $\gamma = 2$ (on a weekly time scale).

%\pagebreak
\vspace{0.75in}
\hline
\vspace{0.25in}
\section*{Data Availability}
The raw data underlying the results presented in the study are openly available to researchers from the {Safegraph} {Patterns} and {Places} datasets found at \url{https://www.safegraph.com/covid-19-data-consortium}. The data generated by our study, including the indoor seasonality metric, is available for download at \url{https://github.com/bansallab/indoor_outdoor}.

\section*{Competing Interests}
The authors declare that they have no competing interests.

\section*{Acknowledgments}
Research reported in this publication was supported by the National Institute of General Medical Sciences of the National Institutes of Health under award number R01GM123007. The content is solely the responsibility of the authors and does not necessarily represent the official views of the National Institutes of Health.

We gratefully acknowledge data sharing by Safegraph which made this study possible. We thank Alexes Merritt for her data processing efforts.

\bibliographystyle{unsrt}
\bibliography{references}

%%%%%%%%%%%%%%%%%%%%%%%%%% SUPPLEMENT
\renewcommand{\thefigure}{S\arabic{figure}}
\renewcommand{\thetable}{S\arabic{table}}
\setcounter{figure}{0}    
\setcounter{table}{0} 
\setcounter{section}{0}
\pagebreak

\section*{Supplementary Figures}

\begin{figure}[h!]
    \centering
    \includegraphics[width=0.95\textwidth]{Figure_S_contact.pdf}
    \caption{(A) Total activity from the Safegraph dataset, centered around the mean, does not vary seasonally in 2018 or 2019. However, there is a marked decrease in activity in early 2020 at the beginning of the COVID-19 pandemic.; (B) Non-household contact data from Facebook CTIS surveys display variability during the COVID-19 pandemic with contacts minimized in April 2020, peaking in the summer, and decreasing in fall 2020. Both plots show data for a random sample of 100 US counties.}
    \label{fig:contactnotseasonal}
\end{figure}

\begin{figure}[h!]
    \centering
    \includegraphics[width=0.95\textwidth]{map_inpropoutpropmean.png}
    \caption{The mean proportion of indoor/outdoor activity in 2018 displays no spatial structure and is relatively homogeneous across counties; outliers  of mean $\geq$ 2.5 are removed}
    \label{fig:indooroutdoormean}
\end{figure}


\begin{figure}[h!]
    \centering
    \includegraphics[width=0.95\textwidth]{FigureSupp_IECC.pdf}
    \caption{(A) The IECC climate zones are based on temperature, humidity, and rainfall in each county and govern the type building material and amount of ventilation required in a building. (B) The consistency between the two primary clusters of indoor activity identified by our analysis and the IECC climate zones.}
    \label{fig:iecc}
\end{figure}

\begin{figure}[h!]
    \centering
    \includegraphics[width=0.95\textwidth]{skimap.png}
    \caption{The third indoor seasonality cluster displays some correlation with areas of increased winter tourism, including US ski areas in western and northeastern states, potentially contributing to off-season activity increases.}
    \label{fig:ski}
\end{figure}

\begin{figure}[h!]
    \centering
    \includegraphics[width=0.95\textwidth]{FigureSupp_2020clustermap.pdf}
    \caption{(A) Indoor seasonality during 2020 can be clustered into four groups, although clusters are more geographically fragmented than previous years. (B) Time series for 2020 indoor seasonality clusters display heterogeneous trends that were not apparent in previous years, with some clusters more variable than others.}
    \label{fig:2020clustermap}
\end{figure}

\begin{figure}[h!]
    \centering
    \includegraphics[width=0.5\textwidth]{beta_functions.png}
    \caption{The seasonal forcing functions ($\beta9t)$) we used in the epidemiological model. The non-seasonal model (grey) shows no variation in transmission risk over time. We model northern seasonality via a sinusoidal model fit to the northern indoor activity data (light blue solid) and via the empirically-measured indoor seasonality from a county in the northern cluster (Cook County, light blue dotted). We model southern seasonality via a sinusoidal model fit to the southern indoor activity data (dark blue solid) and via the empirically-measured indoor seasonality from a county in the northern cluster (Maricopa County, dark blue dotted).}
    \label{fig:betafunction}
\end{figure}

\begin{figure}[h!]
    \centering
    \includegraphics[width=0.5\textwidth]{FigureSupp_sine.pdf}
    \caption{Inferred parameters for the sinusoidal model fits of the indoor activity data for the northern and southern clusters show a similar frequency, but greater amplitude and shorter phase in the northern cluster. Values displayed are mean parameter estimates. Standard errors for all parameters are smaller than 5e-3 and thus are not displayed.}
    \label{fig:sineparamters}
\end{figure}

%\begin{figure}[h!]
%    \centering
%    \includegraphics[width=0.95\textwidth]{FigureSupp_pandemicRt.pdf}
%    \caption{Modeled $R_t$ without including indoor contact clearly differs from observed $R_t$; modeled $R_t$ decreases over time in 2020, while observed $R_t$ peaks in the winter %months and reaches higher magnitudes. With the inclusion of indoor contact, observed $R_t$ is more consistent with modeled $R_t$, both peaking in winter months and reaching similar %magnitudes. }
%    \label{fig:pandemicRt}
%\end{figure}


\end{document}
